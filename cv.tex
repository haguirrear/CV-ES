\documentclass[a4paper,10pt]{article}

%A Few Useful Packages
\usepackage{marvosym}
\usepackage{fontspec} 					%for loading fonts
\usepackage{xunicode,xltxtra,url,parskip} 	%other packages for formatting
\RequirePackage{color,graphicx}
\usepackage[usenames,dvipsnames]{xcolor}
\usepackage[big]{layaureo} 				%better formatting of the A4 page
% an alternative to Layaureo can be ** \usepackage{fullpage} **
\usepackage{supertabular} 				%for Grades
\usepackage{titlesec}	
\usepackage{array}
\usepackage{multirow}
\usepackage{tikz}
\usepackage{graphicx,calc}
\usepackage{longtable}

%% ICONS

\newlength\myheight
\newlength\mydepth
\settototalheight\myheight{Xygp}
\settodepth\mydepth{Xygp}
\setlength\fboxsep{0pt}
\newcommand*\inlinegraphics[1]{%
  \settototalheight\myheight{Xygp}%
  \settodepth\mydepth{Xygp}%
  \raisebox{-\mydepth}{\includegraphics[height=\myheight]{#1}}%
}

\newcommand{\personalLogo}{\includegraphics[height=1.2em]{Icons/personal.png}\hspace{3mm}}%
\newcommand{\introductionLogo}{\includegraphics[height=1.4em]{Icons/reseña.png}\hspace{3mm}}%
\newcommand{\educationLogo}{\includegraphics[height=1.3em]{Icons/education.png}\hspace{3mm}}%
\newcommand{\skillsLogo}{\includegraphics[height=1.6em]{Icons/skills.png}\hspace{3mm}}%
\newcommand{\languagesLogo}{\includegraphics[height=1.4em]{Icons/languages.png}\hspace{3mm}}%
\newcommand{\workLogo}{\includegraphics[height=1.4em]{Icons/work.png}\hspace{3mm}}%
\newcommand{\computerLogo}{\includegraphics[height=1.6em]{Icons/computer.png}\hspace{3mm}}%
\newcommand{\linkedin}{\includegraphics[height=1.4em]{Icons/linkedin.png}\hspace{3mm}}%
\newcommand{\courseLogo}{\includegraphics[height=1.4em]{Icons/course.png}\hspace{3mm}}%
\newcommand{\prizeLogo}{\includegraphics[height=1.4em]{Icons/prize.png}\hspace{3mm}}%


%\newcommand{\Linkedin}{\includegraphics[scale=0.02]{Icons/linkedin.png}}%



%Setup hyperref package, and colours for links
\usepackage{hyperref}
\definecolor{linkcolour}{rgb}{0,0.2,0.6}
\hypersetup{colorlinks,breaklinks,urlcolor=linkcolour, linkcolor=linkcolour}

%FONTS
\defaultfontfeatures{Mapping=tex-text}
%\setmainfont[SmallCapsFont = Fontin SmallCaps]{Fontin}
%%% modified for Karol Kozioł for ShareLaTeX use
\setmainfont[
SmallCapsFont = Fontin-SmallCaps.otf,
BoldFont = Fontin-Bold.otf,
ItalicFont = Fontin-Italic.otf
]
{Fontin.otf}
%%%

%CV Sections inspired by: 
%http://stefano.italians.nl/archives/26
\titleformat{\section}{\Large\scshape\raggedright}{}{0em}{}[\vspace{-1.4em}\hspace{2em}\titlerule]
\titlespacing{\section}{0pt}{3pt}{3pt}
%Tweak a bit the top margin
%\addtolength{\voffset}{-1.3cm}

%Italian hyphenation for the word: ''corporations''
%\hyphenation{im-pre-se}

%-------------WATERMARK TEST [**not part of a CV**]---------------
\usepackage[absolute]{textpos}

\setlength{\TPHorizModule}{30mm}
\setlength{\TPVertModule}{\TPHorizModule}
\textblockorigin{2mm}{0.65\paperheight}
\setlength{\parindent}{0pt}

%--------------------BEGIN DOCUMENT----------------------
\begin{document}

%WATERMARK TEST [**not part of a CV**]---------------
%\font\wm=''Baskerville:color=787878'' at 8pt
%\font\wmweb=''Baskerville:color=FF1493'' at 8pt
%{\wm 
%	\begin{textblock}{1}(0,0)
%		\rotatebox{-90}{\parbox{500mm}{
%			Typeset by Alessandro Plasmati with \XeTeX\  \today\ for 
%			{\wmweb \href{http://www.aleplasmati.comuv.com}{aleplasmati.comuv.com}}
%		}
%	}
%	\end{textblock}
%}

\pagestyle{empty} % non-numbered pages

\font\fb=''[cmr10]'' %for use with \LaTeX command

%--------------------TITLE-------------
\par{\centering
		{\Huge  \textsc{ Hector David Aguirre Arista}
	}\bigskip\par}

%--------------------SECTIONS-----------------------------------
%Section: Personal Data
\section{ \raisebox{-.3\baselineskip}{\personalLogo}\textsc{Personal Information}}

\begin{tabular}{rl}
  \textsc{Place and date of birth:} & \textsc{Cajamarca - Peru  | 22th March 1996} \\
  \textsc{Address:} & \textsc{Jr. Apomayta 532 - San Juan de Lurigancho - Lima} \\
  \textsc{Tel:} & +51 979277722\\
  \textsc{email:} & \href{mailto:hector.aguirre@pucp.pe}{hector.aguirre@pucp.pe}\\
  \textsc{LinkedIn:} & \href{https://www.linkedin.com/in/haguirrear/}{www.linkedin.com/in/haguirrear}\\
\end{tabular}

\section{\raisebox{-.3\baselineskip}{\introductionLogo}Profile}
Mechatronics Engineer from the Pontifical Catholic University of Peru. Backend developer interested also in Machine Learning, Data Science and Internet of Things. Speaks Spanish (mother tongue), English and German. Also with knowledge about programming languages (C, C++, Python, Dart and Javascript), Business Analytics Software (Power BI) and CAD Software (Inventor, Fusion 360, AutoCAD and Eagle). Fond of photography, reading and sports.

%Section: Work Experience at the top
\section{\raisebox{-.3\baselineskip}{\workLogo}Professional Experience}
\begin{longtable}{r|p{11cm}}

  \emph{Aug 2021 - Nov 2021} & Tech Lead\\& \emph{Crehana}\\ & \footnotesize{The integrations between Crehana and other platforms took on more importance after the need of taking Crehana's courses into other applications and courses from external platforms into Crehana. Therefore, an Integrations team was created and I assumed the role of Tech Lead.  Within this role, I carried on projects to develop the integrations in a scalable way and navigate between all platforms easily. One of the key features was the implementation of an SSO process using \emph{SAML2} that allowed Crehana to behave as \emph{IDP} and \emph{SP}. I also used \emph{ElasticSearch} to be able to search quickly between the content of all the external platforms and \emph{serverless architecture} to synchronize the content. In this role, I also had the chance to guide my team in the development of several projects that also included the improvement of the way of work and the reduction of technical debt such as improvements in \emph{CI/CD pipelines}, and the implementation of \emph{GraphQL Federation} and \emph{Microservices Architecture}, etc.}\\&\\
  \emph{Aug 2021 - Nov 2021} & Backend Developer\\& \emph{Crehana}\\ & \footnotesize{Crehana is one of the leading EdTech startups in Latin America. The first project assigned to me was the integration between \emph{SAP Success Factors} (SAP's Learning Management System) and Crehana. In order to achieve this, I deployed an \emph{AWS Step Function} that kept in sync the courses from Crehana into SAP SF. Also, to send the course progress from Crehana to SAP SF, I created events that triggered a \emph{Lambda function} that used SAP's APIs. The events used AWS's \emph{SNS} and \emph{SQS} (with a \emph{DLQ}) to ensure that all messages were processed correctly. I used \emph{Python}, \emph{PostgreSQL}, \emph{Flask}, \emph{GraphQL} and \emph{AWS} in this project.}\\&\\
  \emph{Jun 2021 - Aug 2021} & Backend Engineer.\\& \emph{MINSA (Peru's Ministry of Health)}\\ & \footnotesize{One of the main MINSA's technological projects is "Teleatiendo", a virtual healthcare app, where patients can make medical appointments and doctors can upload prescriptions. Teleatiendo was built using \emph{Django (Python)}, \emph{PostgreSQL} as database and deployed on MINSA's on-premises servers. While working on this project I developed several features that improved the experience on the platform, but the main one allowed doctors to sign medical prescriptions and orders using their electronic identity cards. To build this feature I used \emph{Python, PostgreSQL} and \emph{Nginx} in the backend and \emph{Javascript} on the client side.}\\&\\
  \emph{Mar 2021 - Jul 2021} & Backend Engineer\\& \emph{La Positiva - Digital Transformation Lab}\\ & \footnotesize{La Positiva is one of the biggest insurance companies in Perú. I worked on several projects that aimed to make access to all the products from La Positiva easier for brokers and end customers. One of these projects was "Mobility", a mobile app where I developed several backend features using \emph{ Django (Python)} and \emph{AWS}}\\&\\
  \emph{Dic 2019 - Feb 2021} & Backend Developer \\&\emph{Simple Peru}\\&\footnotesize{Simple Peru is a company that builds tailored techonological solutions (Web applications, Rest APIs, mobile apps and IoT projects). While working here I have developed several REST APIs using \emph{Python, Flask and PostgreSQL} and also a mobile app for Android and iOS using \emph{Flutter}. I was also responsible for the deployment and architectural design of the Backend Solutions. For that i used \emph{Docker} and the cloud services of \emph{AWS}.}\\&\\
  \emph{Dic 2020 - Mar 2021} & Development of a Mobile Training Application about COVID for medical staff located in Peru's jungle native communities\\& \emph{GIZ (German Corporation for International Cooperation GmbH)}\\ & \footnotesize{GIZ is a German development agency that provides services in the field of international development cooperation. One of the projects of GIZ was to provide a Mobile Training Application about COVID to MINSA (Peru's Ministry of Health), to train and inform the medical staff located in the jungle's native communities. This app includes information about how to manage the pandemic in the context of these native communities. My responsibilities were to develop and deploy the backend that the app consumed. I used \emph{Python} and \emph{FastAPI} as framework, it was deployed in \emph{AWS} using \emph{Docker} containers with a proxy (\emph{Nginx}) and \emph{PostgreSQL} as database. Besides, I also developed the mobile application. For that, I used \emph{Flutter} as framework and synchronization and caching techniques, Since the app was meant to work mainly in offline mode.}\\&\\
  \emph{Dic 2019 - Feb2021} & Backend Developer \\&\emph{Simple Peru}\\&\footnotesize{Simple Peru is a company that builds tailored technological solutions (Web applications, Rest APIs, mobile apps and IoT projects). While working here I have developed several REST APIs using \emph{Python, Flask and PostgreSQL} and also a mobile app for Android and iOS using \emph{Flutter}. I was also responsible for the deployment and architectural design of the Backend Solutions. For that i used \emph{Docker} and the cloud services of \emph{AWS}.}\\&\\
  \emph{Feb 2019 - Jul 2019} & Trainee Supplier Development Engineer \\&\emph{American Glass Products}\\&\footnotesize{AGP is a company that produces and designs laminated glass for the main automotive companies of the world, such as Tesla, McLaren, Volkswagen and BMW. Here I had the responsibility to manage and develop the suppliers that the company has. In order to achieve that,  I learned about the Automotive Core Tools (Quality Management tools for the Automotive industry) and about Business Intelligence using Power BI. Moreover, I also developed apps using Power Apps and Flow (Microsoft services)  that automated certain processes within the company.}\\&\\
  \\&\\
  \emph{Apr 2018 - Dic 2018} & Trainee in the Experimental Economics Laboratory \\&\emph{\textsc{Pontifical Catholic University of Peru}}\\&\footnotesize{The Experimental Economics Laboratory (LEEX in Spanish) carried on experiments on research related to decision-making. These experiments were conducted on a computer by diverse people in the form of a game. They were programmed using python. Working here I acquired experience using python and in the management of web pages and servers using Amazon Web Services (AWS).}\\&\\
  \emph{Jan 2017 - Jul 2017} & Trainee \emph{Programmer} in \\&\emph{\textsc{Novatronic}}\\&\footnotesize{Novatronic is a company specialized in software development and transactional solutions. Working there, I had the chance to participate actively in several software development projects. I developed programs using C, C\# and SQL. Moreover, I took part also in the management of these projects, where I was responsible for carried them on correctly.}\\&\\
 \emph{Mar 2016 - Dic 2016} & Professor's Assistant in the \emph{Simulation Tools} lecture of the \\&\emph{\textsc{Pontifical Catholic University of Peru}}\\&\footnotesize{The simulation tools lecture aims  to teach Mechatronic Engineering students to model 3D objects in software as Inventor an Ansys and to build dynamic simulations and apply Finite Elements Analysis to several machines and structures.
 }\\&\\
\end{longtable}


%Section: Education
\section{\raisebox{-.3\baselineskip}{\educationLogo}Studies}
\begin{tabular}{r|l}	
\emph{\textsc{Mar 2013 - Dic 2018}} & \textbf{PONTIFICAL CATHOLIC UNIVERSITY OF PERU} \\
 & \emph{\textsc{Mechatronics Engineering}}\\ 
& \footnotesize{{Graduated in the top tenth of the students}}\\
\\
\emph{\textsc{Oct 2017 - Mar 2018}} & \textbf{UNIVERSIT{\"A}T DUISBURG-ESSEN} \\
 & \emph{Elektrotechnik und Informationstechnik} \\
 & \footnotesize{{Exchange Program (Germany)}} \\
 \multicolumn{2}{c}{}
\end{tabular}

% Skills
% \section{\raisebox{-.3\baselineskip}{\skillsLogo}Skills}
% \begin{itemize}
% 	\item \emph{\textsc{Leadership:}} \\I was \textbf{student representative} on the General Sciences Faculty Council of the PUCP, where I carried on projects, which led to the improvement of the Faculty in its curriculum and infrastructure.  
% 	\item \emph{\textsc{Proactividad:}}\\ I was part of the \textbf{Student Project Management Section} (SEDIPRO in Spanish). In which I carried on several projects to promote good practices in project management.
% 	\item \emph{\textsc{Responsibility:}}\\ I graduated from the PUCP in the 7th place.
% 	\item \emph{\textsc{Communication:}}\\ I worked as Professor's Assistant for a year in the lecture of \emph{Simulation Tools}. A lecture of the Mechatronics Engineering program where it is taught to model 3D solids using Inventor and also to build dynamic simulations and Finite Element Analysis of several machines or structures.
% \end{itemize}

%Section: Languages
\section{\raisebox{-.3\baselineskip}{\languagesLogo}Languages}
\bgroup
\def\arraystretch{1.3}%  1 is the default, change whatever you need
\begin{tabular}{r
>{\raggedright\arraybackslash}p{11cm}}
\textsc{Spanish:}&Mother tongue\\
\textsc{English:}&Advanced\newline \footnotesize{\textsc{Idiomas Católica}}\\
\textsc{German:}&B2/1 level completed \newline \footnotesize{\textsc{Goethe Institut, lima}} \newline \footnotesize{6 months of exchange program in Germany} \\
\end{tabular}
\egroup




\section{\raisebox{-.3\baselineskip}{\computerLogo}Software}


\bgroup
\def\arraystretch{1.3}%  1 is the default, change whatever you need
\begin{tabular}%
{>{\raggedleft\arraybackslash}p{3cm}%
>{\raggedright\arraybackslash}p{11cm}%
}
\textsc{Programming Languages} & \textbullet \hspace{0.5em}\textit{Python:} Implementation of machine learning and Artificial \hspace*{0.7em} Intelligence algorithms and REST and GraphQL APIs using \emph{Flask}, \emph{FastAPI} and \emph{Django}\\
 & \textbullet \hspace{0.5em}\textit{C\# :} Applications with GUI development.\\
& \textbullet \hspace{0.5em}\textit{C:} APIs development and embedded systems programming. \\
& \textbullet \hspace{0.5em}\textit{C++ :} Applications with GUI and APIs development.\\
& \textbullet \hspace{0.5em}\textit{Dart:} Apps development with \textit{Flutter}. \\
& \textbullet \hspace{0.5em}\textit{Javascript:} Simple REST APIs using \emph{Node JS} and Basic frontend using \emph{Vue}. \\
\end{tabular}
\vspace{1em}

\begin{tabular}%
{>{\raggedleft\arraybackslash}p{3cm}%
>{\raggedright\arraybackslash}p{11cm}}
%\textbullet \hspace{0.5em}\textit{Matlab:} Nivel Intermedio. Con experiencia en procesamiento de \hspace*{0.95em}señales e imágenes  y diseño de sistemas de control. \\
\multirow{2}{3cm}{\raggedleft\textsc{Software knowledge}} 
& \textbullet \hspace{0.5em}\textit{Linux based Systems}\\
& \textbullet \hspace{0.5em}\textit{SQL:} PostgreSQL, MySQL.\\
& \textbullet \hspace{0.5em}\textit{Git:} Used in software development projects.\\
& \textbullet \hspace{0.5em}\textit{Docker:} Dockerize applications, usage in CI/CD pipelines and deploy to production and development environments.\\
& \textbullet \hspace{0.5em}\textit{AWS:} Deployment of backend services and usage of their products (S3, ECR, EC2, SQS, SNS, Step Functions, Api Gateway, etc.).\\
& \textbullet \hspace{0.5em}\textit{ElasticSearch:} Search engine development.\\
& \textbullet \hspace{0.5em}\textit{Inventor:} 3D objects and technical drawings design. \\
& \textbullet \hspace{0.5em}\textit{Fusion 360:} 3D objects and technical drawings design. \\
& \textbullet \hspace{0.5em}\textit{Eagle:} Electronic boards design.\\
& \textbullet \hspace{0.5em}\LaTeX: Writing academic documents.\\
%& \textbullet \hspace{0.5em}\textit{Ladder:} Nivel Intermedio.\\
\end{tabular}
\egroup


\section{\raisebox{-.3\baselineskip}{\courseLogo}Courses}

\bgroup
\def\arraystretch{1.5}%  1 is the default, change whatever you need
\begin{tabular}{
>{\raggedleft\arraybackslash}p{3cm}%
>{\raggedright\arraybackslash}p{11cm}}
 \textsc{Self Organizing Embedded Systems}  & Lecture from the Embedded Systems Master \newline \footnotesize{\textsc{Universit\"at Duisburg-Essen (Germany)}}\\
 \textsc{Information Mining} & Lecture from the Applied Computing Master \newline \footnotesize{\textsc{Universit\"at Duisburg-Essen (Germany)}}\\
\end{tabular}
\egroup

\section{\raisebox{-.3\baselineskip}{\prizeLogo}Awards and Achievements}
\bgroup
\def\arraystretch{1.1}%  1 is the default, change whatever you need
\begin{tabular}{
>{\raggedleft\arraybackslash}p{3cm}%
>{\raggedright\arraybackslash}p{11cm}}
 \multirow{2}{3cm}{\raggedleft\textsc{First Place} \break Jun 2019}  & 5th Hackathon organized by the \textit{National Society of Mining, Oil and Energy} (Peru) and by the Entrepreneurial Development Center of the ESAN University \\
 & \footnotesize{Solution Dtech: Distributed system for real-time air quality monitoring and early warnings for critical events.}\\ & \footnotesize{\raisebox{-.3\baselineskip}{\includegraphics[height=1.4em]{Icons/link.png}} \href{https://perumin.com/perumin34/notas-de-prensa/estudiantes-de-la-pucp-crean-sistema-de-alerta-ante-accidentes-en-minas-subterraneas}{https://perumin.com/perumin34/notas-de-prensa/estudiantes-de-la-pucp-crean-sistema-de-alerta-ante-accidentes-en-minas-subterraneas}}\\
 
\end{tabular}
\egroup

%Section: Scholarships and additional info
%\section{Scholarships and Certificates}
%\begin{tabular}{rl}
% \textsc{Sept.} 2006 & Scholarship for graduate students with an outstanding curriculum \footnotesize(\EURcr 30,000)\normalsize\\
%\textsc{June} 2006 & {\textsc{Gmat}\textregistered}\setmainfont[SmallCapsFont=Fontin-SmallCaps.otf]{Fontin.otf}: 730 (\textsc{q:50;v:39}) 96\textsuperscript{th} percentile; \textsc{awa}: 6.0/6.0 (89\textsuperscript{th} percentile)
%\end{tabular}

%\section{Intereses y Actividades}
%Technology, Open-Source, Programming\\
%Paradoxes in Decision Making, Psychoanalysis, Behavioural Finance\\
%, Travelling


%\newpage
%\hypertarget{gmat}{\textsc{Gmat}\setmainfont{LMRoman10 Regular}\textregistered\setmainfont[SmallCapsFont=Fontin-SmallCaps]{Fontin-Regular}}

%\XeTeXpdffile ''GMAT.pdf'' page 1 scaled 800

\end{document}
